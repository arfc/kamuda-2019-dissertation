\chapter{Uranium Enrichment Regression Results and Discussion}

\section{Problem Description and Training Dataset Overview}



\section{Results from Training all Models}

% Is it necessary to re-run the hyperparameter search, or are the 'simple' models good enough?

Asymptotic models are 


Again, measure asymptotic MSE convergence.



\section{Results from Adding Physics}

Show how accuracy changes when using
\begin{itemize}
    \item Multiple detector-source distance
    \subitem CNN, DNN, BS-DAE, BS-CAE, V-DAE, BS-CAE
    \item Cement floor
    \subitem CNN, DNN, BS-DAE, BS-CAE, V-DAE, BS-CAE
    \item Solid vs shell vs solid and shell uranium
    \subitem CNN, DNN, BS-DAE, BS-CAE, V-DAE, BS-CAE
    \item 
\end{itemize}

\section{Results from Changing Energy Windows}

The energy windows that are important for uranium enrichment measurements are below 300 keV. Because of this, we can adjust our algorithm to use only these energy windows (Explain why the FWHM of these peaks are fine to use). The downside of this is we're not looking at the 1460 keV or 2614 keV background peaks for calibration. Will the background subtracting autoencoder deal with this? Will it's performance decrease? Can the autoencoder still subtract background in a useful way?

Show how accuracy changes when using
\begin{itemize}
    \item Full spectrum
    \subitem CNN, DNN, BS-DAE, BS-CAE, V-DAE, BS-CAE
    \item Reduced ROI
    \subitem CNN, DNN, BS-DAE, BS-CAE, V-DAE, BS-CAE
\end{itemize}






