Current radioisotope identification devices perform poorly. There is a need to develop an algorithm that can identify and quantify isotopes in low-resolution gamma-ray spectra in a wide range of realistic conditions. Trained gamma-ray spectroscopists typically rely on intuition when identifying isotopes in spectra. Pattern recognition algorithms such as neural networks are prime candidates for automated isotope identification. A trained gamma-ray spectroscopists can inject intuition into these algorithms by creating training datasets and choosing machine learning model for a task. Algorithms based on feature extraction such as peak finding or ROI algorithms work well for well-calibrated high resolution detectors. For low-resolution detectors, it may be more beneficial to use algorithms that incorporate more abstract features of the spectrum. To investigate this, dense, convolutional, and autoencoder artificial neural networks (ANNs) were trained to perform identification and quantification tasks using gamma-ray spectra. Datasets are simulated and used to train the ANNs for each task. Because datasets are simulated, this method can be extended to different gamma-ray spectroscopy tasks. In this work we introduce \verb|annsa|, an open source Python package capable of creating gamma-ray spectroscopy training datasets and applying machine learning models to solve spectroscopic tasks. In this work we demonstrate \verb|annsa|'s capabilities on a source interdiction classification and uranium enrichment quantification problem.